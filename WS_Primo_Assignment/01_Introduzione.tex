\section{Introduzione}
L'elevata abbondanza dei \emph{Fast Food} ed il loro abuso da parte dei consumatori ha conseguenze sulla loro salute a causa dell'elevato apporto calorico che contribuisce in maniera determinante all'aumento di malattia potenzialmente letali. Fornire informazioni nutrizionali ha un grosso impatto sulle decisione dei consumatori, permettendogli di autoregolarsi e promuovere diete più sane e, quindi, molti governi hanno imposto la pubblicazione dei contenuti nutrizionali per assistere i consumatori. Tuttavia, le informazioni nutrizionali dei fast food sono frammentate e non facilmente leggibili da parte della maggior parte dei consumatori, è quindi necessario raccogliere i dati nutrizionali per sintetizzare e semplificare le conoscenze.

E' stata sviluppata un'ontologia incentrata sui Fast Food e, più in generale, sulle informazioni nutrizionali che riguardano i loro prodotti, questa ontologia rappresenta un tentativo di standardizzazione della conoscenza dei fast food tentando di collegare i dati nutrizionali che sono aggregati ed analizzati per esigenze informative di consumatori ed esperti. L'ontologia si basa sui metadati di 21 risorse nutrizionali di fast food ed è stata creata in OWL2 utilizzando Protégé.

La logica dell'ontologia è stata valutata attraverso la traduzione in linguaggio naturale, sebbene la maggioranza concordava sulla veridicità dell'ontologia, sono stati identificate un piccolo gruppo di affermazioni errate, dunque è stata rivisitata e pubblicata ufficialmente la versione corretta. L'ontologia ha 413 classi, 21 proprietà degli oggetti, 13 proprietà dei dati e 494 assiomi logici.

Con il rilascio iniziale dell'ontologia si sono anche discusse visioni future legate ad essa, soprattutto per lo sviluppo e l'evoluzione della stessa (riportate in fondo a questo report), la più interessante riguarda la gestione e la pubblicazione di voluminose quantità di dati nutrizionali dei fast food collegati tra di loro semanticamente.
