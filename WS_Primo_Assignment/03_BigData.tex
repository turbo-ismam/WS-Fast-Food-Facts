\section{Big Data e ontologie correlate}

Le informazioni e i fatti nutrizionali (come anche detto nelle sezioni precedenti) possono essere presentati ai consumatori come fattore per indicare la salubrità di un certo prodotto e può essere parte integrante nella decisione fatta per effettuare scelte alimentari sane. 
Data la quantità di fast food, oltre ai ristoranti non fast food, i consumatori hanno una varietà di opzioni per fare scelte potenzialmente nutrizionali e dietetiche. 

Il lavoro dietro la OFFF è incentrata su come gestire il volume dei dati nutrizionali dei fast food e metodi per raccogliere la grande quantità di dati per renderli disponibili e interrogabili. 

Il focus di \emph{Fast Food Facts} è esclusivamente incentrato nel dominio del fast food, tuttavia ci sono alcune ontologie alimentari "generali" che si sono utilizzate per capire come organizzare Fast Food Facts, tra le più importanti:
\begin{itemize}
    \item \textbf{FoodOn:} un'ontologia basata su BFO (formalontologia di base) che ha un'ampia base di conoscenze che copre vari aspetti della conoscenza alimentare, inclusa l'origine agricola dei singoli prodotti alimentari
    \item \textbf{Food Ontology:} è un'ontologia applicativa di base delle ontologie per cibo e ricette della \emph{British Broadcasting Corporation}
    \item \textbf{The FoodOntology Knowledge Base:} è un modello base di informazioni nutrizionali sugli alimenti presente nel database del Ministero dell'Alimentazione della Turchia.
    \item \textbf{Open Food Facts:} è una fonte terminologica gratuita e crowdsourcing di prodotti alimentari internazionali che si affida a volontari, tuttavia si presume che possa essere soggetto a potenziali errori a causa del suo approccio crowd-sourced e manca di qualsiasi meccanismo per verificare le informazioni.
\end{itemize}

Fondamentalmente, questo è un argomento BigData che condivide alcune sue caratteristiche (velocità, volume e varietà). A causa della domanda del mercato, ci si aspetterebbe che i prodotti fast food (insieme alle informazioni nutrizionali) cambino e aumentino anche con più scelte dai singoli ristoranti e dai locali emergenti (caratteristiche di velocità e volume dei BigData). 
Inoltre, OFFF assume che con una quantità limitata di informazioni nutrizionali presentate, se i dati nutrizionali fossero collegati ad altre fonti di dati esterne, la conoscienza aumenterebbe ulteriormente. Per molti ristoranti, le informazioni nutrizionali sono presentate in formati variabili: statici e dinamici siti Web, download PDF, siti Web in silos, ecc., e sono disponibili anche opzioni di menu regionali per ospitare un segmento della popolazione mondiale, ma senza una soluzione accessibile per aggregare le informazioni per l'analisi e il processo decisionale. 

OFFF ha l'obiettivo di facilitare il collegamento dei dati alle informazioni nutrizionali e fornire metodi per effettuare interrogazioni attraverso le fonti nutrizionali eterogenee di fast food. 

\emph{Abstractlythis} genera un grafico di rete di informazioni di dominio dai collegamenti relazionali tra i concetti. Le tecnologie semantiche come OWL2 e RDF supportano la creazione delle ontologie e una sintassi basata su macchine per condividere e interpretare la conoscenza standardizzata di un dominio. 

Nell'ambito delle funzionalità dei Big Data, le ontologie affrontano la varietà dei dati attraverso la standardizzazione e la normalizzazione di fonti di dati eterogenee e il collegamento ad altre fonti, la velocità con la flessibilità di modificare lo schema per adattarsi alla rapida crescita dei dati e anche il volume attraverso tecnologie del web semantico come le nanopubblicazioni. 

OFFF si basa sulla struttura delle informazioni nutrizionali opensource incentrate sul consumatore presentate sui siti Web di fast food. Oltre a formalizzare e avere una concettualizzazione condivisa delle conoscenze nutrizionali dei fast food, la disponibilità di questa base di conoscenze contribuisce a casi d'uso futuri che possono potenzialmente avvantaggiare consumatori sanitari e ricercatori e clinici di classe esperta.



