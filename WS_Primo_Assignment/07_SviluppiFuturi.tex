\section{Conclusione}
Le cattive abitudini alimentari dovute all'eccesso di cibo o alle scelte alimentari hanno un impatto sull'insorgenza di gravi malattie potenzialmente letali e hanno un impatto sulla qualità della vita dei pazienti.
Esistono ricerche che alludono al fatto che la presentazione di informazioni nutrizionali potrebbe avere un impatto sul comportamento alimentare sano. Tuttavia, l'abbondante quantità di informazioni nutrizionali, insieme a potenziali dati estesi, complica qualsiasi sforzo per aggregare e centralizzare le informazioni nutrizionali che i consumatori e gli esperti possono utilizzare. 
Questa ontologia mira a normalizzare e standardizzare fonti di dati eterogenee di informazioni sui fast food , e facilitare un volume elevato e una quantità di dati nutrizionali fast food in rapida evoluzione. 
L'ontologia è stata rivista per valutare la costruzione logica del livello concettuale poligerarchico dei dati nutrizionali dei fast food e dei fast food. 
Questa ontologia è un primo passo verso una potenziale direzione futura per realizzare completamente una fonte pubblica persistente e interrogabile di dati nutrizionali incentrati sul consumatore collegati di fast food.