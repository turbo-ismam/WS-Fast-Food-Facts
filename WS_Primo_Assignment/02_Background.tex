\section{Background}
I fast food forniscono cibi prodotti in \emph{serie} e serviti rapidamente. Le offerte di menu più salutari includono insalate e carni magre alla griglia, ma rimangono pizza, hamburger e patatine fritte e questi ultimi sono gli articoli più comunemente acquistati. 
Negli ultimi 3 decenni, gli americani hanno aumentato l'assunzione di cibi preparati fuori casa e attualmente, il 36\% degli adulti statunitensi mangia ogni giorno nei fast-food.  
I fast food sono popolari tra le persone della maggior parte dei gruppi di età a causa del loro basso costo, consistenza, convenienza e reperibilità. La maggior parte delle ricerche mostra che il consumo di fast food è legato all'aumento di peso in eccesso, alla scarsa qualità della dieta e alla mortalità.

\subsection{Impatto dei fast food sulla salute}
Il cibo è essenziale per ogni essere umano ed esso ha un impatto significativo sulla salute di un individuo, quindi una dieta equilibrata avrà un impatto positivo sulla salute di un individuo mentre una dieta sbilanciata avrà effetti negativi.
Il consumo di fast food può portare ad un aumento di peso eccessivo, per verificare in maniera oggettiva questo fatto si ricorre all'indice di massa corporea (\emph{IMC}), questo è un dato attraverso cui si può classificare un individuo (sottopeso, normopeso, sovrappeso). L'aumento di peso in eccesso mette le persone a rischio di sviluppare malattie e condizioni che aumentano la probabilità di morte a causa di malattie cardiovascolari.
Alcune ricerche suggeriscono che vivere in aree densamente popolate da fast food può avere un impatto sulla salute individuale a causa della maggiore accessibilità dei fast food per il consumo.

I risultati complessivi degli studi indicano che esiste un'associazione tra il consumo di fast food e l'aumento di peso in eccesso. Uno studio prospettico di coorte condotto da su 541 bambini in età prescolare ha rilevato che lo stato di peso aumentava nei bambini che consumavano fast food più frequentemente durante la settimana.

Invece i risultati condotti da svariati esperimenti hanno esaminato una serie di articoli per determinare l'associazione dell'ambiente alimentare della zona con il peso corporeo delle persone, in tutti questi studi è evidente il fatto che la densità dei fast food in una determinata zona va ad influire negativamente sul peso corporeo delle persone che frequentano determinata zona, gli studi presi in considerazione sono quelli di William et al., Mazidi et al. ed un'analisi ecologica mirata ad esaminare più nello specifico la densità dei fast food e la prevalenza del tipo 2 tra le contee della Carolina del Sud.
Da questi studi si può dedurre che l'ambiente alimentare incide sulla salute degli individui.

Esiste un'associazione tra il consumo di fast food e la qualità della dieta, vari studi, come la valutazione di Barnes, lo studio trasversale di Vercammen e lo studio di Todd et al. illustrano questa relazione.
In alcuni casi però si erano notati alcuni dati che non tornavano, in quanto mostravano che l'assunzione di grassi saturi e colesterolo stava diminuendo, si è poi scoperto che in realtà era dettato dal fatto che era la qualità dei fast food stessi ad essere migliorata.
Si pensa che l'etichettatura dei menu migliora la capacità dei consumatori di riconoscere i prodotti alimentari a basso consumo energetico e suggeriscono che ciò potrebbe costringere i ristoranti a riformulare le loro voci di menu abbassando il loro contenuto energetico.

In sintesi, possiamo affermare che il peso eccessivo di un individuo in molti casi è proporzionale all'abuso nel consumo dei fast food, la densità degli stessi è proporzionale al loro uso da parte delle persone ed in fine che esiste una relazione tra consumo di fast food e la qualità della dieta delle persone.

\subsection{Impatto delle informazioni nutrizionali sulla salute}
L'accesso alle informazioni nutrizionali influisce sul modo in cui le persone gestiscono la propria salute attraverso la dieta. Gli individui con malattie croniche possono monitorare l'assunzione di nutrienti come sodio e zucchero per rallentare la progressione della malattia, mentre quelli senza malattie croniche possono utilizzare le informazioni nutrizionali per la prevenzione delle malattie. Le informazioni nutrizionali influenzano anche le decisioni relative agli alimenti che le persone scelgono di acquistare e mangiare.

Alcuni studi hanno preso on considerazione la relazione tra comportamenti alimentari sani e l'uso delle etichette alimentari, i risultati ottenuti suggeriscono che le persone che usano le etichette per scegliere avevano maggiori probabilità di fare scelte alimentari sane, come consumare più frutta e verdura e meno bibite. Sfruttando il fatto delle etichette è poi possibile giungere a una serie di vantaggi che dipendono da una dieta sana ed equilibrata, come ricorrere meno a malattie cardiovascolari ecc.
Sono quindi emerse prove che indicano che gli individui con disturbi alimentari e problemi di peso possono essere notevolmente influenzati dall'esposizione all'etichettatura dei menu.
