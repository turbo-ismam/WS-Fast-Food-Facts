\section{Risultati}
L'ontologia Fast Food Facts (OFFF) Contiene 413 classi, 21 proprietà degli oggetti, 13 proprietà dei dati e 494 assiomi logici. I tre valutatori hanno annotato in modo indipendente 430 affermazioni in linguaggio naturale. E' stato annotato ciascuna affermazione in termini di:
\begin{itemize}
    \item (0) se l'affermazione non è stata espressa accuratamente dall'ontologia
    \item (1) se l'affermazione è stata espressa accuratamente dall'ontologia
\end{itemize}
Le affermazioni che hanno suscitato una risposta di "non so" sono state annotate come (0). I valutatori hanno raggiunto una notevole affidabilità dell'intercodificatore con una concordanza percentuale media a coppie del 76,1\%. 
L'accordo a coppie era del 73\% per i valutatori 1 e 2, dell'84,7\% per i valutatori 2 e 3 e del 70,7\% per i valutatori 1 e 3. 
L'affidabilità dell'intercoder è stata calcolata utilizzando ReCal3 0,1 Alpha per 3+ codificatori. 
Nella media totale tra i valutatori, i valutatori hanno valutato che il l'ontologia ha rappresentato un numero accurato di affermazioni del 73,0\% (media del 56,5\%, 81,2\% e 81,2\%). 
Sono stati presi in considerazione 103 affermazioni in tutti questi criteri e sono stati esaminati i problemi con queste affermazioni per valutare la fonte dell'inesattezza di OFFF. 
Si è notato tre tipi di problemi con l'accuratezza di OFFF: 
\begin{enumerate}
    \item scarsa etichettatura delle entità che avrebbero potuto beneficiare di etichette più elaborate per una migliore espressione dell'assioma
    \item etichettatura errata che esprimeva informazioni che non riflettevano il mondo (ad es. Moka $\subset$ Coffee\_Fact, "ogni moka è un Coffee Fact")
    \item errori logici e possibile confusione e contesa in cui il problema non era l'etichetta ma un problema con l'associazione dell'affermazione che potrebbe aver portato a confusione o incomprensione (es. DQ\_Treatzza\_Pizza $\subset$ Cake, “ogni DQ Tratzza Pizza è un Cake”).
\end{enumerate}
Per il terzo caso, c'erano 24 affermazioni che rientravano in quella categoria. Per il primo e il secondo caso, c'erano rispettivamente 48 e 31 affermazioni. E' stata rivista l'ontologia sulla base di questi disaccordi maggioritari sulla veridicità. 
Il modello di questa ontologia ha rappresentato fatti poligerarchici di fast food invece di rappresentarli come entità di cibo. Si è orientato il modello ontologico per riflettere una sana organizzazione del fast food, e questo si riflette nel progetto di cui si è discusso in precedenza. 
Con problemi che ruotano attorno a una maggiore elaborazione delle etichette, sono state aggiunte etichette più espressive aggiungendo termini, come la modifica di Taco $\subset$ Insalata ("ogni taco è un'insalata") a Taco\_Salad $\subset$ Insalata ($\sim$ "ogni insalata di taco è un'insalata"). 
Per il terzo numero c'era una combinazione di errori di scrittura, come affermazioni duplicate ma errate, e incomprensione dell'accuratezza dell'affermazione. 
Per gli errori di creazione, si ha avuto un'affermazione come Mushroom\_Swiss\_Burger $\subset$ Mushroom ("ogni hamburger svizzero di funghi è un fungo"), che sono stati successivamente eliminati a causa di un duplicato e di un errore. 
C'erano affermazioni che si basavano su una conoscenza particolare dell'alimento di alcuni ristoranti (come Dairy Queen e Whataburger) che utilizzavano la propria nomenclatura (ad esempio, Apple\_Bites $\subset$ Apple\_Slices o Hash\_Brown\_Sticks\_Whataburgers). 
Anche per il terzo numero, c'erano affermazioni che avrebbero potuto essere accurate a seconda della precedente comprensione della definizione dei concetti. 
Ad esempio, OFFF ha Hot\_Dog come tipo di Fast\_Food\_Sandwich, se si dovesse capire che un panino è un alimento impanato con un qualche tipo di ripieno non di pane. 

L'ultima versione di questa ontologia è disponibile al link del repository git, https://github.com/UTHealth-Ontology/OFFF.